%!TEX root = ../msc17-game-book.tex

\phChapter{About the Game}

  Congratulations on your acceptance as a new research team for television's
  (second-)best-known show on the history of museums,
  \textbf{Enigmas of the Exhibits}. As you know, each episode features
  mysteries hidden within museum exhibits around the world.

  This week, we feature the
  \textbf{National Museum of Mathematics}, or \textbf{MoMath} for
  short. Its motto, \textit{Ums Okdriv Ep Favvil Gsxxdv}, describes MoMath's
  vision of promoting mathematics with its excellent exhibits on
  concepts such as fractals, game theory, and probability.
  (Or so we assume; our Latin is a bit rough.) But what we need
  you to uncover is the whereabouts of MoMath's mysterious founder,
  \textbf{Professor W. Fayes}. As you know, the professor
  recently vanished without
  a trace, but we have reason to believe that he left clues to his
  disappearance within the exhibits and puzzles he left behind...

\phSection{Schedule}

\begin{itemize}
\item Registration Opens | 4:00pm
\item Orientation | 4:30pm
\item Puzzling Begins | 4:45pm
\item Solutions Due | 8:00pm
\item Wrap-up Presentation | 8:00pm
\item Q\&A and Awards | 8:30pm
\item Dismissal | 9:00pm
\end{itemize}

\phSection{Casual/Expert Teams}

Teams may participate at either the Casual or Expert level, and are ranked
separately. Expert Teams will receive more cryptic variations of the Main
Puzzles and Metapuzzle. Victory Points for Bonus Puzzles are awarded
separately for Casual and Expert Teams.

\phSection{Opening Puzzle}

Before you can investigate the mysteries of Professor W. Fayes's puzzles,
your team will first need to solve a puzzle which involves the
exhibits themselves. Move quickly, as the sooner you solve this opening
challenge, the sooner you can start working on the rest of the hunt!

\phSection{Main Puzzles}

After solving the Opening Puzzle, your team will receive a packet of five
Main Puzzles. Work with your teammates to decode each of these to a
short phrase. For each correct solution you submit to Game Control,
your team will earn \textbf{15 Victory Points}. Each solution also unlocks a
Bonus Puzzle, so make sure to pick it up from Game Control!

\phSection{Hint Tokens}

Hint Tokens are hidden throughout the museum: a small sticker with the
MoMath logo and a codeword. If your team needs a nudge, you can report
a Hint Token codeword to Game Control for a hint. Each codeword may only
be used once by each team.

\phSection{Bonus Puzzles}

Each Bonus Puzzle is an extension of the Main Puzzle which unlocked it.
Your team must find the optimal solution to the given challenge, and
the team(s) submitting the best solution for each Bonus Puzzle
to Game Control during competition will receive \textbf{10 Victory Points}.
In the case of a tie, all tied teams will receive 10 Victory Points.

Game Control will grade solutions as they are submitted. Teams may
try to improve their solution; however, each submission replaces
any previous ones (even if a previous submission was better). Each team
is limited to three submissions per Bonus Puzzle.

\phSection{Metapuzzle}

Your main goal is to find out where Professor W. Fayes disappeared to!
Reporting his current location to Game Control will net your team a whopping
\textbf{20 Victory Points}.

\phSection{Another Puzzle?}

You never know! It seems that \textbf{15 Victory Points} not accounted for above
are hidden away somewhere... If you think you've solved another puzzle hidden
by the professor, show your work to Game Control before solutions are due.

\phSection{Endgame}

All solutions are due at Game Control by 8:00pm;
players not in line to submit a solution will be turned away after that time.

\phSection{Winning the Game}

For the mathematicans in the audience:
teams are ranked by a lexicographic ordering of \((V,-H,M,-T)\), where
\(V\) is the number of Victory Points earned, \(H\) is the number of hints
taken, \(M\) is the number of Main Puzzles solved, and \(T\) is the timestamp
for the final Main Puzzle solved.

Following is a less silly explanation.
The team which has earned the most Victory Points by the end of the competition
is the winner. If two teams are tied for the same number of Victory Points,
then the team which took fewer Hints during the game will be ranked higher.
If that does not break the tie, the team which solved more Main Puzzles
will be ranked higher. If that still doesn't break the tie, the team which
solved their last Main Puzzle earliest will be ranked higher.

\phSection{Rules}

\begin{itemize}
\item Players should not do anything which would interfere with other teams
      playing the game.
\item Players may not use personal electronic devices to assist their teams
      during the competition.
\item Have fun and maybe learn some math while puzzling!
\end{itemize}

\phSection{Attribution}

This hunt was designed by Dr. Steven Clontz, Assistant Professor of Mathematics
at The University of South Alabama. \phUrl{http://clontz.org}
Dr. Clontz serves as Director of Mathematical Puzzle Programs (MaPP),
and has also designed several other puzzle games in the southeast United States.
MaPP runs mathematical puzzlehunts
for high school students in partnership with several colleges and universities
across the country.

Several puzzles in this event were adapted from MaPP events
under the Creative Commons Attribution 4.0 license; see \phUrl{http://mappmath.org}
for more information. Dr. Clontz would like to thank the following designers
whose MaPP puzzles were adapted for this event: Jeffery Ford
(Bonus Puzzle 1), Kelly Bragan Guest (Main Puzzle 3), and Amy Steinkampf
(Main Puzzle 4).
