%!TEX root = ../../msc17-game-book.tex


\phWorksheet{Bonus Puzzle 4}

Count Calcula now has to face his scariest task yet: meeting
with a certain Ms. Knope of the Transylvania Parks and Recreation office.

According to city code, there must be exactly one park in each area
of the city which may be walked to without entering a building or
crossing a road. For example, if Batty Borough had the same code, the Count
would need to build \(8\) parks:


\begin{center}
\begin{tikzpicture}[x=0.3in,y=0.2in]
\phDrawBattyBorough{}
\node at (-5,-1) {\large\bf 1};
\node at (-6,3) {\large\bf 2};
\node at (-3,3) {\large\bf 3};
\node at (0,4) {\large\bf 4};
\node at (3,3) {\large\bf 5};
\node at (6,3) {\large\bf 6};
\node at (5,-1) {\large\bf 7};
\node at (0,-4) {\large\bf 8};
\end{tikzpicture}
\end{center}

Remember, Transylvania has \textbf{57 buildings} and \textbf{73 roads},
and most importantly, no bridges (so roads cannot cross one another).
Believe it or not, it's possible to predict for Ms. Knope how many parks
should be in Transylvania. Can you figure out the pattern? (Here's a hint:
try drawing some small towns and adding up the numbers of buildings, roads,
and parks... do you see a pattern?)

\textbf{How many parks should there be in Transylvania?}

\vspace{2em}

\textbf{Mark your answer below.}

\begin{itemize}
  \item[\Huge\(\circ\)] \(11\)
  \item[\Huge\(\circ\)] \(18\)
  \item[\Huge\(\circ\)] \(27\)
  \item[\Huge\(\circ\)] \(39\)
  \item[\Huge\(\circ\)] \(56\)
\end{itemize}
