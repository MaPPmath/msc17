%!TEX root = ../../msc17-game-book.tex


\phWorksheet{Bonus Puzzle 1}

Now Grendel has bewitched a broom to do the hard work of stirring her
caudlrons for her, as shown below.

\begin{center}
\begin{tikzpicture}[x=0.3in,y=0.3in]
  \draw[color=gray,very thin,fill=lightgray] (0,0) circle (3);
  \draw[color=gray,very thin,fill=lightgray] (6,0) circle (3);
  \draw[color=gray,very thin,fill=lightgray] ($(3,{3*sqrt(3)})$) circle (3);
  \draw (0,0) -- ($(3,{3/sqrt(3)})$) -- (3,0) -- (0,0);
  \node at (1.7,-0.4) {\small 3};
  \node at (1.6,1.4) {\small \(2\sqrt{3}\)};
  \node at (3.3,0.7) {\small \(\sqrt{3}\)};
  \draw[color=gray,very thin,fill=lightgray] ($(3,{3/sqrt(3)})$) circle (0.1);
\end{tikzpicture}
\end{center}

Magiked brooms are rather finicky however, and unlike Grendel won't work
in too tight a space. So, \textbf{can you figure out the area of the small
space between the three cauldrons}? If it helps, the area of a circle is
\(\pi r^2\) where \(r\) is the radius of the circle, and the area of
a triangle is \(\frac{1}{2}bh\) where \(b\) is the length of its base
and \(h\) is the height of the triangle measured perpendicularly to the base.

\vspace{2em}

\textbf{Mark your answer below.}

\begin{itemize}
  \item[\Huge\(\circ\)] \(2\pi\sqrt3\)
  \item[\Huge\(\circ\)] \(9\sqrt3-\frac{9}{2}\pi\)
  \item[\Huge\(\circ\)] \(4\sqrt6+3\pi\)
  \item[\Huge\(\circ\)] \(6\pi\sqrt6\)
  \item[\Huge\(\circ\)] \(\pi\sqrt3+\frac{9\sqrt3}{4}\)
\end{itemize}
% seven cauldron image
% \begin{center}
% \begin{tikzpicture}[x=0.2in,y=0.2in]
%   \draw[color=gray,very thin,fill=lightgray] (0,0) circle (3);
%   \draw[color=gray,very thin,fill=lightgray] (6,0) circle (3);
%   \draw[color=gray,very thin,fill=lightgray] (-6,0) circle (3);
%   \draw[color=gray,very thin,fill=lightgray] ($(3,{3*sqrt(3)})$) circle (3);
%   \draw[color=gray,very thin,fill=lightgray] ($(-3,{3*sqrt(3)})$) circle (3);
%   \draw[color=gray,very thin,fill=lightgray] ($(3,{-3*sqrt(3)})$) circle (3);
%   \draw[color=gray,very thin,fill=lightgray] ($(-3,{-3*sqrt(3)})$) circle (3);
%   \draw (0,0) -- ($(3,{3/sqrt(3)})$) -- (3,0) -- (0,0);
%   \node at (1.7,-0.4) {\tiny 3};
%   \node at (1.6,1.4) {\tiny \(2\sqrt{3}\)};
%   \node at (3.3,0.7) {\tiny \(\sqrt{3}\)};
%   \draw[color=gray,very thin,fill=lightgray] ($(3,{3/sqrt(3)})$) circle (0.1);
%   \draw[color=gray,very thin,fill=lightgray] ($(3,{-3/sqrt(3)})$) circle (0.1);
%   \draw[color=gray,very thin,fill=lightgray] ($(-3,{3/sqrt(3)})$) circle (0.1);
%   \draw[color=gray,very thin,fill=lightgray] ($(-3,{-3/sqrt(3)})$) circle (0.1);
%   \draw[color=gray,very thin,fill=lightgray] ($(0,{2*sqrt(3)})$) circle (0.1);
%   \draw[color=gray,very thin,fill=lightgray] ($(0,{-2*sqrt(3)})$) circle (0.1);
% \end{tikzpicture}
% \end{center}
